%%%%%%%%%%%%%%%%%%%%%%%%%%%%%%%%%%%%%%%%%
% Medium Length Professional CV
% LaTeX Template
% Version 2.0 (8/5/13)
%
% This template has been downloaded from:
% http://www.LaTeXTemplates.com
%
% Original author:
% Trey Hunner (http://www.treyhunner.com/)
%
% Important note:
% This template requires the resume.cls file to be in the same directory as the
% .tex file. The resume.cls file provides the resume style used for structuring the
% document.
%
%%%%%%%%%%%%%%%%%%%%%%%%%%%%%%%%%%%%%%%%%

%----------------------------------------------------------------------------------------
%	PACKAGES AND OTHER DOCUMENT CONFIGURATIONS
%----------------------------------------------------------------------------------------

\documentclass{resume} % Use the custom resume.cls style

\usepackage[left=0.5in,top=0.5in,right=0.5in,bottom=0.5in]{geometry} % Document margins
\usepackage{hyperref}
\usepackage[T1]{fontenc}
\usepackage{tgpagella}
\usepackage{dtklogos}
\name{Hongyi Xia} % Your name
\address{11048 Amherst Ave. \\ Silver Spring, MD 20902} % Your address
\address{(301)~$\cdot$~541~$\cdot$~7635 \\ hongyi.xia@gmail.com \\ \url{www.linkedin.com/in/hongyixia}} % Your phone number and email

\begin{document}

%----------------------------------------------------------------------------------------
%	EDUCATION SECTION
%----------------------------------------------------------------------------------------

\begin{rSection}{Education}

{\bf University of Maryland, College Park} \hfill { Expected: May 2015} \\
B.S. in Aerospace Engineering - Aerospace Engineering Honors Program \hfill {GPA: 3.75} \\
Coursework: Aerodynamics, Dynamics, Controls, Vibrations, Thermodynamcis, Heat Transfer, Propulsion, Aerospace Structures\\
Hinman CEOs Entrepreneurship Program ~$\cdot$ University Honors Program ~$\cdot$  Alpha Omega Epsilon Professional Engineering Sorority ~$\cdot$ Tau Beta Pi National Engineering Honor Society ~$\cdot$Sigma Gamma Tau National Aerospace Honor Society\\
Math Success Program - Tutored students in undergraduate math courses. ~$\cdot$SEEDS Peer Mentoring Program \\
Teaching Fellow for ENES232 Thermodynamics. \\
\end{rSection}

%----------------------------------------------------------------------------------------
%	WORK EXPERIENCE SECTION
%----------------------------------------------------------------------------------------

\begin{rSection}{Experience}
\begin{rSubsection}{Space Exploration Technologies}{September 2013 - August 2014}{Propulsion Intern}{Hawthorne, CA}
\item Developed software to visualize 3D combustion simulations of the Raptor rocket engine in C++ and OpenGL. Integrated with Leap Motion sensor for interaction via hand gestures.
\item CFD and thermal analysis of the Merlin 1D rocket engine, Crew Dragon vehicle, Falcon 9 v1.1 vehicle, and Falcon 9 v1.1 developmental vehicle in ANSYS CFX, Star-CCM+, and Thermal Desktop to aid design, ensure correct operation of various components, and to keep on schedule for flight vehicle development and production.
\item Created algorithms to automate data review of the Merlin 1D and Merlin Vacuum engine. Complete rewrite of current engine performance code to calculate engine parameters of single engine acceptance test data and stage test data for improved readability, accuracy, and consistency. Built an understanding of the engine fluid system, core rocket equations, pump maps, stage calculations etc.
\item Supported daily operations of the Liquid Engine Development team in data review and flight data acquisition during launch operations. Developed application to support flight data acquisition.
\end{rSubsection}

%------------------------------------------------

% \begin{rSubsection}{Interaction Lab - University of Southern California}{June - August 2013}{Undergraduate Researcher}{Los Angeles, CA}
% \item Developed exercises for the NAO robot, an exercise buddy for overweight children in an obesity intervention study. Participated in the Viterbi Summer Undergraduate Research Experience (SURE) program.
% \end{rSubsection}

%------------------------------------------------
\begin{rSubsection}{Collective Dynamics and Control Laboratory}{June 2010 - May 2013}{Undergraduate Researcher}{College Park, MD}
\item Designed and performed flow visualization experiments to validate a potential flow model. Research culminated in paper
	titled ``Experimental Flow Visualization of Karman Vortex Flow past a Fish-inspired Robot''. Presented at the 2013 AIAA
	Region I-MA Student Conference. (2012-2013)
\item   Developed a method to visualize 3D flows using Microsoft Kinect as a sensor. Calibrated and programmed Microsoft Kinect cameras in OpenCV and Python to collect depth video data. Wrote Matlab scripts to image process depth video data of smoke rings and calculate their flow properties. Participated in the National Science Foundation sponsored Miniature Robotics Research Experiences for Undergraduates Program at the University of Maryland (2012)
\item Wrote Matlab simulations to model the dynamics of rigid body collisions to study collective behavior of hexbugs. (2011)
\item Developed a motion coordination algorithm for Lego NXT that enables two robotic tanks to balance on a seesaw. (2010)
\end{rSubsection}

%------------------------------------------------

% \begin{rSubsection}{Institute for Systems Research}{June - August 2012}{Undergraduate Researcher}{College Park, MD}
% \item Developed a method to visualize 3D flows using Microsoft Kinect as a sensor. Calibrated and programmed Microsoft Kinect cameras in OpenCV and python to collect depth video data. Wrote Matlab scripts to image process depth video data of smoke rings and calculate their flow properties.
% \item Participated in National Science Foundation sponsored Research Experiences for Undergraduates in Miniature Robotics. Program activities included laboratory tours, guest speakers on robotics research, graduate school mentoring, and culminating research paper and presentation.
% \end{rSubsection}
\end{rSection}


%----------------------------------------------------------------------------------------
%	Leadership Experience
%----------------------------------------------------------------------------------------

% \begin{rSection}{Leadership Experience}
% \begin{rSubsection}{Math Success Program}{September 2012 - May 2013}{Math Coach}{College Park, MD}
% \item Tutored students in undergraduate math courses.
% \end{rSubsection}

% \begin{rSubsection}{ENES232 Thermodynamics}{January - May 2012}{Teaching Fellow}{College Park, MD}
% \item Facilitated recitation for 60 students. Graded student quizzes. Held weekly office hours.
% \end{rSubsection}

% \begin{rSubsection}{SEEDS Peer Mentoring Program}{September 2011 - May 2013}{Peer Mentor}{College Park, MD}
% \item Mentored a group of freshman engineering students to facilitate their adjustment to college and the engineering program.
% \end{rSubsection}

% \end{rSection}

%----------------------------------------------------------------------------------------
%	SKILLS
%----------------------------------------------------------------------------------------

\begin{rSection}{Skills}

\begin{tabular}{ @{} >{\bfseries}l @{\hspace{6ex}} l }
Languages & Matlab, Java, Python, C++, OpenGL, git, \LaTeX \\
Applications & Ansys, Star-CCM+, Thermal Desktop, Autodesk Inventor, Siemens NX, Visual Studio \\
& Linux, Mac OS X, Windows \\
\end{tabular} \\
Mandarin Chinese (native speaker)
\end{rSection}


%----------------------------------------------------------------------------------------
%	AWARDS SECTION
%----------------------------------------------------------------------------------------

\begin{rSection}{Awards}
1st Place, JP Morgan Code for Good Challenge - Delaware \hfill{September 2014}\\
Banneker Key Scholar - Full, four year scholarship to the University of Maryland.  \hfill { March 2010}   \\
Siemens Competition in Math, Science \& Technology Semifinalist \hfill {October 2009} \\
Maryland Distinguished Scholar Finalist (Academic Achievement) \hfill {October 2009}  \\
National AP Scholar \hfill {August 2009}  \\

\end{rSection}
%----------------------------------------------------------------------------------------

\end{document}
